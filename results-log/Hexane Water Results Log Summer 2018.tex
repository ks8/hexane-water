% LaTEX source code
% Last modified November 1st, 2005
% Steve Miller
% note that the percent sign comments out the rest of the line
% first, we set a document class. often use 12pt characters, though
% sometimes people do 11 or 10. you can do report or article, both similar
%\documentclass[12pt,letterpaper]{article}
\documentclass[12pt,reqno]{amsart}
\linespread{1}
\addtolength{\textwidth}{2cm} \addtolength{\hoffset}{-1cm}
\addtolength{\marginparwidth}{-1cm} \addtolength{\textheight}{2cm}
\addtolength{\voffset}{-1cm}
% below are some packages that are needed for certain symbols, graphics, colors.
% safest to just include these.
\usepackage{times}
\usepackage[T1]{fontenc}
\usepackage{mathrsfs}
\usepackage{latexsym}
\usepackage[dvips]{graphics}
\usepackage{epsfig}
\usepackage{hyperref, amsmath, amsthm, amsfonts, amscd, flafter,epsf}
\usepackage{amsmath,amsfonts,amsthm,amssymb,amscd}
\input amssym.def
\input amssym.tex
\usepackage{color}
\usepackage{enumerate}
\usepackage{hyperref}
\usepackage{url}
\usepackage{floatrow}
\usepackage{caption}
\usepackage{subcaption}
\usepackage{capt-of}
\usepackage{physics}
\newcommand{\todo}[1]{\textcolor{red}{\textbf{(#1)}}}

    %=======================================================

    %   THIS IS WHERE YOU PUT SHORTCUT DEFINITIONS

    %========================================================

% Note that we use a percent sign to comment out a line

% below are shortcut commands

%%%%%%%%%%%%%%%%%%%%%%%%%%%%%%%%%%%%%%%%%%%%%%%

% below are shortcuts for equation, eqnarray,

% itemize and enumerate environments

\newcommand\be{\begin{equation}}
\newcommand\ee{\end{equation}}
\newcommand\bea{\begin{eqnarray}}
\newcommand\eea{\end{eqnarray}}
\newcommand\bi{\begin{itemize}}
\newcommand\ei{\end{itemize}}
\newcommand\ben{\begin{enumerate}}
\newcommand\een{\end{enumerate}}
\newcommand{\ncr}[2]{\left({#1 \atop #2}\right)}
%%%%%%%%%%%%%%%%%%%%%%%%%%%%%%%%%%%%%%%%%%%%%%%%

% Theorem / Lemmas et cetera

\newtheorem{thm}{Theorem}[section]
\newtheorem{conj}[thm]{Conjecture}
\newtheorem{cor}[thm]{Corollary}
\newtheorem{lem}[thm]{Lemma}
\newtheorem{prop}[thm]{Proposition}
\newtheorem{exa}[thm]{Example}
\newtheorem{defi}[thm]{Definition}
\newtheorem{exe}[thm]{Exercise}
\newtheorem{rek}[thm]{Remark}
\newtheorem{que}[thm]{Question}
\newtheorem{prob}[thm]{Problem}
\newtheorem{cla}[thm]{Claim}
\newtheorem{defis}[thm]{Definitions}
\newtheorem{res}[thm]{Result}
\newtheorem{calc}[thm]{Calculation}
%%%%%%%%%%%%%%%%%%%%%%%%%%%%%%%%%%%%%%%%%

% shortcuts to environments

% this allows you to do textboldface: simply type \tbf{what you want in bold}

\newcommand{\tbf}[1]{\textbf{#1}}

%%%%%%%%%%%%%%%%%%%%%%%%%%%%%%%%%%%%%%%%%%%%%%%%%%

% shortcut to twocase and threecase definitions

\newcommand{\twocase}[5]{#1 \begin{cases} #2 & \text{#3}\\ #4
&\text{#5} \end{cases}   }
\newcommand{\threecase}[7]{#1 \begin{cases} #2 &
\text{#3}\\ #4 &\text{#5}\\ #6 &\text{#7} \end{cases}   }
%%%%%%%%%%%%%%%%%%%%%%%%%%%%%%%%%%%%%%%%%

%Blackboard Letters

\newcommand{\R}{\ensuremath{\mathbb{R}}}
\newcommand{\C}{\ensuremath{\mathbb{C}}}
\newcommand{\Z}{\ensuremath{\mathbb{Z}}}
\newcommand{\Q}{\mathbb{Q}}
\newcommand{\N}{\mathbb{N}}
\newcommand{\F}{\mathbb{F}}
\newcommand{\W}{\mathbb{W}}
\newcommand{\Qoft}{\mathbb{Q}(t)}  %use in linux
\newcommand{\soln}{\noindent \textbf{Solution:}\ }

%%%%%%%%%%%%%%%%%%%%%%%%%%%%%%%%%%%%%%%%%

% Finite Fields and Groups

\newcommand{\Fp}{ \F_p }
%%%%%%%%%%%%%%%%%%%%%%%%%%%%%%%%%%%%%%%%%

% Fractions

\newcommand{\foh}{\frac{1}{2}}  %onehalf
\newcommand{\fot}{\frac{1}{3}}
\newcommand{\fof}{\frac{1}{4}}

%%%%%%%%%%%%%%%%%%%%%%%%%%%%%%%%%%%%%%%%%

% Legendre Symbols

\newcommand{\js}[1]{ { \underline{#1} \choose p} }

%%%%%%%%%%%%%%%%%%%%%%%%%%%%%%%%%%%%%%%%%

% matrix shortcuts

\newcommand{\mattwo}[4]
{\left(\begin{array}{cc}
                        #1  & #2   \\
                        #3 &  #4
                          \end{array}\right) }
\newcommand{\matthree}[9]
{\left(\begin{array}{ccc}
                        #1  & #2 & #3  \\
                        #4 &  #5 & #6 \\
                        #7 &  #8 & #9
                          \end{array}\right) }
\newcommand{\dettwo}[4]
{\left|\begin{array}{cc}
                        #1  & #2   \\
                        #3 &  #4
                          \end{array}\right| }
\newcommand{\detthree}[9]
{\left|\begin{array}{ccc}
                        #1  & #2 & #3  \\
                        #4 &  #5 & #6 \\
                        #7 &  #8 & #9
                          \end{array}\right| }
%%%%%%%%%%%%%%%%%%%%%%%%%%%%%%%%%%%%%%%%%

% greek letter shortcuts

\newcommand{\ga}{\alpha}                  %gives you a greek alpha
\newcommand{\gb}{\beta}
\newcommand{\gep}{\epsilon}
%%%%%%%%%%%%%%%%%%%%%%%%%%%%%%%%%%%%%%%%%

% general functions

\newcommand{\notdiv}{\nmid}               % gives the not divide symbol
\newcommand{\burl}[1]{\textcolor{blue}{\url{#1}}}

%%%%%%%%%%%%%%%%%%%%%%%%%%%%%%%%%%%%%%%%%%%

% the following makes the numbering start with 1 in each section;

% if you want the equations numbered 1 to N (without caring about

% what section you are in, comment out the following line.

\numberwithin{equation}{section}

%\textwidth= 6in

%\evensidemargin=37pt

%\oddsidemargin=0pt

\begin{document}



\title{Hexane Water Results Log Summer 2018 (August-Present)}
\author{Kirk Swanson}
\email{swansonk1@uchicago.edu}
\address{Institute for Molecular Engineering, University of Chicago, 5640 S Ellis Ave, Chicago, IL 60637}
\date{\today}




\maketitle

%%%%%%%%%%%%%%%%%%%%%%%%%%%%%%%%%%%%%%%%%%%%%%%%%%%%%%%%%%%%%%%%%%%%%%%%%%%%%%%%%%%%%%%%%%%%%%%%%%%%%%%%%%%%%%%%%%%%%%%%%%%%%%

\normalsize

%%%%%%%%%%%%%%%%%%%%%%%%%%%%%%%%%%%%%%%%%%%%%%%%%%%%%%%%%%%%%%%%%%%%%%%%%%%%%%%%%%%%%%%%%%%%%%%%%%%%%%%%%%%%%%%%%%%%%%%%%%%%%%
\section{8/6/2018}
\begin{enumerate}
\item Downloaded pressure\_scalar-press-NPT-1bead-water-lammps to dash\_work/interface on laptop.  Used pressure\_analysis\_UPDATE.py leaving out first 2000 records and used 198 blocks 5 ps to get average pressure of 1.41, SE 0.94, SD 511.08: 
\begin{figure}[H]
\centering
\includegraphics[scale=0.4]{pressures_press-NPT-1bead-water-lammps}
\end{figure}
\item Because we accidentally used restart/traj every 1,000 for fulldata-NPT-298 and then restart/traj 10,000 for fulldata-NPT-298-restart1, we will re-do this calculation again using the updated interface file collecting restart/traj every 1,000.  
\item Submitted simulation no. 1.  
\item To check pressures in a non-vacuum interfacial system, in DASH we are going to run simulation no. 2, which does NPT equilibration for 200,000 steps and then does an NVT simulation of the interface.  Filename press-nonexpanded.
\item Submitted simulation no. 2.  
\item We want to compare the above simulation to LAMMPS, so we will first run the equilibration step in LAMMPS as simulation no. 3.  Note, however, that the mixing rules currently used in the LAMMPS file are arithmetic.  Also NOTE: to use depablo-tc, must type module load openmpi into terminal first, even if it is already in submit script.  After this, we need to run the NVT portion of the simulation for 6M steps.  
\item Running simulation no. 3.
\item Now, let's turn to how pressure is typically computed in DASH.  In general, an ideal gas mechanical equation of state is given by 
\begin{align}
\begin{split}
P = \frac{Nk_BT}{V} = \rho k_BT
\end{split}
\end{align}
This assumes that molecules in the gas are point-like and do not interact with each other.  However, if interactions between the particles and a finite particle volume is allowed.  The virial expansion is a perturbation theory around the ideal gas used when the interactions in the real gas are dominated by two-body interactions.  One uses a virial expansion in powers of the density, keeping the temperature dependence of the coefficients.  We can then apply manipulations of the time derivative of the classical virial to arrive at 
\begin{align}
\begin{split}
P = \rho k_B T - \left[\frac{1}{3V}\left<\sum_{i<j}^Nr_{ij}F_{ij}\right>\right]
\end{split}
\end{align}
where the second term is called the virial correction.  
\item In DASH, it appears as though DataComputerPressure.cu is responsible for scalar pressure values.  On line 92, we have 
\begin{align}
\begin{split}
pressureScalar = (tempScalar\_loc * ndf\_loc * boltz + sumVirial) / (dim * volume) * state->units.nktv\_to\_press
\end{split}
\end{align}
tempScalar\_loc appears to be the temperature of the system, boltz is the boltzmann coefficient, sumVirial is that second virial correction, dim is the dimension of the system.  ndf I calculated to be 3 per atom, including M sites for the water model (printed values using interface file).  So, it does make sense to divide by dim, i.e. divide by 3, because that gets us to N.  However, we are also including M-sites in the pressure value.  So, we are at the very least overcounting the ideal gas term for the pressure.  The amount of overcorrection for 3650 water molecules is given by:
\begin{align}
\begin{split}
P_{c} = \frac{Nk_BT}{V} = \frac{3650\times1.38\times10^{-23}\times 298}{50 \times 10^{-10}\times 50\times 10^{-10}\times 150\times 10^{-10}}
\end{split}
\end{align}
Actually, would it even be incorrect to have four sites per atom?  Maybe that's actually the correct ideal model here?
\end{enumerate}

\section{8/7/2018}
\begin{enumerate}
\item First, we will begin by compiling the most recent version of DASH, which should contain Mike's edits to the TIP4P/F fix for accurate pressure computation (avoiding M-site in ideal gas term).  Compiling version in folder DASH-8-7-2018.  
\item DASH pressure computations:
\subitem NVT with vacuum: run\_UPDATE\_8-7-2018.sh and interface\_UPDATE\_8-7-2018.py, z length 149.2717, x length 58.45467, y length 58.67825, 3650 TIP4P/F water molecules, 500 hexane molecules, waldman-hagler mixing between the water and hexane molecules, NVT with Andersen thermostat with parameters nu = 0.01 (a parameter describing the collision frequency of the system with the heat bath) and applying every 10 steps, 298 K, 1M steps equilibration and 1M steps production, PI false, restart/traj every 10,000, -20/+20 for z change as accounted above, filename newpress-expanded, tensor information commented out in input script. This is simulation no. 4.   
\subitem We will do the same as above, except we will not expand the box by 20 A in each z direction, and we will apply NPT equilibration for 1M steps and then 1M steps of NVT.  This is simulation no. 6.  

\item LAMMPS pressure computations:
\subitem NVT with vacuum: interface.sbatch and in.taffi\_tip4pF\_waldman, x length 149.2717, y length 58.67825, z length 58.45467, 3650 TIP4P/F water molcules, 500 hexane molecules, waldman-hagler mixing, NVT with NoseHoover, 298 K, 1M steps equilibration and 1M steps production, PI false, restart/traj every 10,000, filename newpress-expanded-lammps.  We will start with 500,000 steps and restart from there, which is simulation no. 5. 
\subitem Same as above, except we will not expand the box by 20 A in each z direction, and we will apply NPT equilibration for 1M steps and then 1M steps of NVT.  The first 500,000 steps will be simulation no. 7.  
\subitem Simulation no. 8 is filename newpress-expanded-lammps-restart1, restarting from newpress-expanded-lammps.restart for 1.5M steps 
\subitem Simulation no. 9 is filename newpress-nonexpanded-lammps-restart1, restarting from newpress-nonexpanded-lammps.restart for 500K steps
\subitem Simulation no. 10 is filename newpress-nonexpanded-lammps-restart2, restarting from newpress-nonexpanded-lammps-restart1.restart, and going for 1M NVT steps.  
\end{enumerate}

\section{8/8/2018}
\begin{enumerate}
\item From simulation no. 8, we use pressure\_analysis\_UPDATE.py on laptop in dash\_work/interface folder to analyze pressures\_scalar-newpress-expanded-lammps-restart.txt.  We use indices 50,000 to 150,000 and 100 SE blocks of size 5 ps.  This gives us a mean pressure of -83.443 atm, SE of the mean 1.578 atm, and SD of the distribution 154.699 atm.  Here is a plot of these values over time:
\begin{figure}[H]
\centering
\includegraphics[scale=0.7]{pressures_newpress-expanded-lammps-restart1}
\end{figure}
\item From simulation no. 4, we use pressure\_analysis\_UPDATE.py on laptop in dash\_work/interface to analyze pressure\_scalar-newpress-expanded.txt.  We use indices 100,000 to 200,000 and 100 SE blocks of size 5 ps.  This gives us a mean pressure of -70.603 atm, SE of the mean 1.487 atm, and SD of the pressures 154.854.  Here is a plot:
\begin{figure}[H]
\centering
\includegraphics[scale=0.7]{pressures_newpress-expanded}
\end{figure}
Removing the first 100 values for better comparison to the above plot, we have:
\begin{figure}[H]
\centering
\includegraphics[scale=0.7]{pressures_newpress-expanded-clipped}
\end{figure}
\item If we combine pressures\_scalar-newpress-expanded-lammps.txt and pressures\_scalar-newpress-expanded-lammps-restart1.txt, then we have all 2M steps covered.  We call this file pressures\_scalar-newpress-expanded-lammps-combined.txt.  Analyzing all points in this file, using 200 blocks of size 5 ps, we get an average pressure of -80.945, SE of the mean 1.237, and SD of values 156.056:
\begin{figure}[H]
\centering
\includegraphics[scale=0.7]{pressures_newpress-expanded-lammps-combined}
\end{figure}
\item Using all values from pressures\_scalar-newpress-expanded.txt from DASH, using 200 blocks as well, we get a mean pressure of -77.716, SE of the mean 1.337, and SD of values 155.610:
\begin{figure}[H]
\centering
\includegraphics[scale=0.7]{pressures_newpress-expanded-full}
\end{figure}
\item Andersen Thermostat: Rescales velocities at regular timesteps to control the temperature
\item NoseHoover Thermostat: Integrates equations of motion derived from a Hamiltonian that has an extra degree of freedom for the heat bath
\item From simulation no. 6, in DASH, we did 1M NPT steps followed by 1M NVT steps so that we can analyze pressure from the NVT which will not have vacuum, in theory.  The box dimensions we get from this simulation are: x length 48.875, y length 49.061, z length 91.3641.  Downloading pressure\_scalar-newpress-nonexpanded.txt to dash\_work/interface on laptop for analysis as above.  Using the 1M NPT plus 1M NVT data and 200 blocks, we have a mean pressure of -31.943, SE of the mean 3.174, and SD of values 333.460:
\begin{figure}[H]
\centering
\includegraphics[scale=0.7]{pressures_newpress-nonexpanded-full}
\end{figure}
Using only the 1M NVT steps, and 100 blocks, we have a mean pressure of -32.501, SE of the mean 3.592, SD of values 364.244:
\begin{figure}[H]
\centering
\includegraphics[scale=0.7]{pressures_newpress-nonexpanded}
\end{figure}
\item From simulation no. 10, in LAMMPS, we did 1M NVT steps after 1M NPT equilibration.  The box dimensions are: x length 90.498, y length 48.597, z length 48.412.  These lengths are slightly different from the above DASH simulation.  The NVT component of the simulation, from pressures\_scalar-newpress-nonexpanded-lammps-restart2.txt, we get a mean pressure of 54.504, SE of the mean 4.47, and SD of values 375.729, using 100 blocks:
\begin{figure}[H]
\centering
\includegraphics[scale=0.7]{pressures_newpress-nonexpanded-lammps-restart2}
\end{figure}
Combining the NPT steps into pressures\_scalar-newpress-nonexpanded-lammps-combined.txt, we get a mean pressure of 25.947, SE 3.152, and SD 374.379 using 200 blocks: 
\begin{figure}[H]
\centering
\includegraphics[scale=0.7]{pressures_newpress-nonexpanded-lammps-combined}
\end{figure}
\item It is possible that we have differing values here because the volumes are slightly different.  To test the hypothesis that the vacuum is responsible for the negative pressure, let's try decreasing the volume by only putting 10 Angstroms on each side of the simulation box.  
\item Simulation no. 11 in DASH will take care of this filename newpress-smallexpanded, and simulation no. 12 in LAMMPS will do the same. 
\item Simulation no. 13 will be DASH 2M NPT of the interface
\item Simulation no. 14 will be LAMMPS NPT of the interface
\item Simulation no. 15 will be DASH 2M NVT at 0.997 of TIP4P/F 3650 molecules
\item Simulation no. 16 will be 2M NPT at 1 atm of 3650 molecules in DASH
\end{enumerate}


\section{8/9/2018}
\begin{enumerate}
\item 
\begin{figure}[H]
\centering
\includegraphics[scale=0.7]{pressures_newpress-NPT}
\end{figure}
\begin{figure}[H]
\centering
\includegraphics[scale=0.7]{pressures_newpress-NPT-lammps}
\end{figure}
\end{enumerate}

\section{8/11/2018}
\begin{enumerate}
\item First, we need to re-do molecular orientation plots.  The file molecular\_orientation\_water\_postproduction\_UPDATE.py contains the unwrap update that is necessary for accurate analysis.  In this file, we correctly unwrap molecules based on the location of the oxygen atom.  In dash\_work/interface on Midway, we will use orientation\_postproduction.sbatch to do water orientation postproduction profiles from the 1bead trajectory, full-298-1bead, and the 32 bead trajectory, full-298-32bead-combined.xyz.  The filename for 1bead is orientations\_water\_unwrap.txt, which once downloaded to laptop was copied as orientations\_water\_full-298-1bead-orientations-water.txt.  The filename for 32bead is orientations\_water\_32bead-water-unwrap.txt.    
\item For full-298-1bead, we used indices 600 to 960:
\begin{figure}[H]
\centering
\includegraphics[scale=0.7]{full-298-1bead-orientations-water-corrected}
\end{figure}
\item We also corrected molecular\_orientation\_hexane\_postproduction\_UPDATE.py to have the appropriate unwrapping procedure.  We use this in orientation\_postproduction\_hexane.sbatch to analyze the same trajectory files as above for hexane.  The 1bead file is orientations\_hexane\_full-298-1bead-orientations-hexane-corrected.txt.  We download this onto laptop and use indices 900 to 1150 to obtain:
\begin{figure}[H]
\centering
\includegraphics[scale=0.7]{full-298-1bead-orientations-hexane-corrected}
\end{figure}
We are also running the 32 bead version, named as above, orientations\_hexane\_full-298-32bead-orientations-hexane-corrected.txt.
\item We recently resolved the negative pressure issue.  Let's start from Section 2.5 of The Molecular Theory of Capillarity, by J. S. Rowlinson and B. Widom.  The assumption we begin with is that it is possible to consistently define local values of the thermodynamic fields $p, T, \mu$ even in an inhomogeneous system.  At a planar fluid-fluid interface, confied to a cube of sidelength l and interfacial area A = $l^2$, these fields and densities are functions only of the z dimension, defined as normal to the interfacial plane.  We call the pressure in either bulk phase the scalar quantity $p$.  Near the interface the pressure is a tensor since its tangential components, parallel with a dividing surface, include the tension of the interface, and so differ from the normal component.  Mechanical stability requires that the gradient of this tensor is zero everywhere in the fluid, $\triangledown \cdot \vec{p} = 0$ , and the symmetry of the surface requires that $\vec{p}$ is a diagonal tensor, $\vec{p}(\vec{r}) = p_{xx}(\vec{r})e_xe_x + p_{yy}(\vec{r})e_ye_y + p_{zz}(\vec{r})e_ze_z$ with $p_{xx}(\vec{r}) = p_{yy}(\vec{r})$.  Substitution of this equation into the gradiant equation yields $p_{zz}(z) = p_{yy}(z) = p_T(z)$ and $p_{zz}(z) = p_N(z) = p$, where $p_N$ and $p_T$ are the normal and transverse components of the pressure.  If a side-wall of the cube is displaced isothermally and reversibly so as to increase the area by $\delta A$, then the tengential work done on the system is $\delta W_T = -\delta A\int_{-l/2}^{l/2}p_T(z)dz$.  A similar displacement of the top or bottom wall by a distance $\delta A / l$ serves to maintain the volume constant, and requires further (normal) work $\delta W_N = A(\delta A/l)p = l\delta Ap$.  Thus, the total work done, and hence the increase in free energy, is $\delta F = \delta W_T + \delta W_N = \delta A\int_{l/2}^{l/2}[p - p_T(z)]dz$.  Using $dF = -SdT - pdV + \gamma dA + \mu dN$.  So, we can associate this integral with the interfacial tension.    
\item This brings us to the following.  We define the interfacial tension as $\gamma = \int_0^{L_z}[p_N(z) - p_T(z)]dz$.  Since at a stable planar interface the gradiant of the pressure tensor must be zero, the tensor must be symmetric, and the bulk pressures have scalar value $p$, we can define $p_N(z) = p_{zz} = p$ and $p_T(z) = \frac{p_{xx}(z) + p_{yy}(z)}{2}$.  Using the mean value theorem, we can write the average pressure tensor elements obtained from an MD simulation, $<p_{xx}> = p_{xx}$ and $<p_{yy}> = p_{yy}$ as $p_{xx} = \frac{1}{L_z}\int_0^{L_z}p_{xx}(z)dz$ and $p_{yy} = \frac{1}{L_z}\int_0^{L_z}p_{yy}(z)dz$.  We can then carry out the integration that defines $\gamma$ and rewrite the interfacial tension as $\gamma = (p - \frac{p_{xx} + p_{yy}}{2})$, which is the standard formula for computing interfacial tension in molecular simulations.  In our simulation box, we have two interfaces, so the interfacial tension associated with a single interface can be given by dividing by two:  $\gamma = \frac{1}{2}(p - \frac{p_{xx} + p_{yy}}{2})$.  We assume that the bulk phases are at their equilibrium densities corresponding to $p \approx 1 \text{ atm }$, either by measuring the bulk densities in an NVT simulation or by applying a barostat in the z dimension to maintain $p \approx 1 \text{ atm }$.  For hexane-water, the experimental interfacial tension is 50 $\frac{mN}{m} = 4, 900 \text{atm}\cdot $\AA.  Our simulation box is approximately 100 \AA in the z dimension, so, we have 4,900 atm $\cdot$ \AA = (1 atm - $\frac{p_{xx} + p_{yy}}{2})\cdot$ 100 \AA.  We can solve this to get $p_{xx} + p_{yy} = -194$.  Since the scalar pressure in an MD simulation is computed as $P_{tot} = \frac{p_{xx} + p_{yy} + p_{zz}}{3}$, we have $P_{tot} = -64.33$.  What happens to these negative scalar pressures in the thermodynamic limit?  In ambient conditions, $\gamma$ would still be 50 $\frac{mN}{m}$ and $p = 1 \text{ atm }$, but $L_z\rightarrow\infty$, so $\text{lim}_{L_z\rightarrow\infty}\frac{\gamma}{L_z} = 0$, which implies that $\frac{p_{xx} + p_{yy}}{2}$ converges to $p$ and $P_{tot} \rightarrow 1$. 
\begin{align}
\begin{split}
\gamma = \int_{0}^{L_z}[p_N(z) - p_T(z)]dz
\end{split}
\end{align} 
\begin{align}
\begin{split}
p_N(z) = p_{zz} = p
\end{split}
\end{align} 
\begin{align}
\begin{split}
p_T(z) = \frac{p_{xx}(z) + p_{yy}(z)}{2}
\end{split}
\end{align} 
\begin{align}
\begin{split}
= \frac{1}{2}\left(p - \frac{p_{xx} + p_{yy}}{2}\right)L_z
\end{split}
\end{align} 
\begin{align}
\begin{split}
p_{xx} = \left<p_{xx}(z)\right> = \frac{1}{L_z}\int_{-L_z/2}^{L_z/2}p_{xx}(z)dz
\end{split}
\end{align} 
\begin{align}
\begin{split}
4,900 \text{ atm } \text{\AA} \approx \frac{1}{2}(1 \text{ atm } - \frac{p_{xx} + p_{yy}}{2})\cdot 100 \text{\AA}
\end{split}
\end{align} 
\begin{align}
\begin{split}
p_{xx} + p_{yy} \approx -194
\end{split}
\end{align} 
\begin{align}
\begin{split}
P_{\text{tot, MD sim.}} = \frac{p_{xx} + p_{yy} + p_{zz}}{3} \approx -65\text{ atm}
\end{split}
\end{align}
\begin{align}
\begin{split}
\text{lim}_{L_z\rightarrow\infty}\frac{\gamma}{L_z}\rightarrow = 0
\end{split}
\end{align} 
\begin{align}
\begin{split}
\frac{p_{xx} + p_{yy}}{2} \rightarrow p
\end{split}
\end{align} 
\end{enumerate}

\section{8/12/2018}
\begin{enumerate}
\item Working on slides explaining capillary theory contributions for candidacy ppt.  
\item We download orientations\_hexane\_full-298-32bead-hexane-orientations-corrected.txt and analyzed:
\begin{figure}[H]
\centering
\includegraphics[scale=0.7]{full-298-32bead-orientations-hexane-corrected}
\end{figure}
\item Download orientations\_water\_32bead-water-unwrap.txt.  Had issues with nan in orientation\_profile\_analysis\_UPDATE for some strange reason, nans in the SE measurements, so used nan\_policy='omit' for stats.sem.  Finally, it works: used zlo 600 zhi 940:
\begin{figure}[H]
\centering
\includegraphics[scale=0.7]{full-298-32bead-orientations-water-corrected}
\end{figure}
\end{enumerate}

\section{8/13/2018}
\begin{enumerate}
\item PIMC barostat acceptance criterion:
\begin{align}
\begin{split}
A(V'/V) = \text{min}[1, e^{-\beta P(V' - V)}e^{-\beta[(U(r') - U(r))/nBeads]}e^{N\text{ln}(V'/V)}]
\end{split}
\end{align}
\item Developing ppt slide on density fluctuations:
\begin{align}
\begin{split}
\rho_W(z) = \frac{1}{2}\rho_W + \frac{1}{2}\rho_W\text{erf}\left(\frac{z - \left<h_W\right>}{\sqrt{2}\sigma_c}\right) \\
\rho_H(z) = \frac{1}{2}\rho_H - \frac{1}{2}\rho_H\text{erf}\left(\frac{z - \left<h_H\right>}{\sqrt{2}\sigma_c}\right) 
\end{split}
\end{align}
\begin{align}
\begin{split}
\sigma_0 = \left<h_H\right> - \left<h_W\right> \\
\sigma_c^2 = \frac{k_BT}{2\pi\gamma}\text{ln}\left(\frac{L_{||}}{l_b}\right) 
\end{split}
\end{align}
\end{enumerate}

\section{8/14/2018}
\begin{enumerate}
\item 
\begin{align}
\begin{split}
\le
\end{split}
\end{align}
\end{enumerate}

\section{8/23/2018}
\begin{enumerate}
\item Test 1: P\_16 on midway-l13b-21, printing to test.xyz but not output right away.  About 175,000 printings to .xyz (about 10,000 steps) but none to output.   Started printing to output after about 4:00 minutes.  Stopped printing and said "Turn 3271000 15.86 per"  without finishing line.  At about 8 minutes it printed again to output file, catching up...and immediately got stuck again.  
\item Test2: P\_16 on midway2-gpu06, printing to output right away, no problem. 
\item Test 3: P\_16 on midway-l13b-21, printing to test.xyz but not output right away.   
\end{enumerate}

\section{9/26/2018}
\begin{enumerate}
\item According to Professor de Pablo, the priority item for the path integral research project is to fix the path integral monte carlo barostat in DASH.  So, we will start by investigating the code that Mike has already written for this barostat on the DASH github.
\item First, let's compile the most recent version of DASH on Midway.  Doing "git clone https://github.com/dreid1991/md\_engine.git".  Saving this compiled version on Midway 1 in DASH-9-26-2018.  
\item Information on BoostPython: https://wiki.python.org/moin/boost.python/GettingStarted.  
\subitem The BoostPython library binds C++ and Python in a mostly-seamless fashion.  It is just one member of the boost C++ library collection.  Use the BoostPython library to quickly and easily export C++ to python such that the python interface is very similar to the C++ interface.  Boost.Python bindings are written in pure C++, using no tools other than your editor and your C++ compiler.  
\end{enumerate}

\section{10/1/2018}
\begin{enumerate}
\item Created tip4pF\_9-26-2018.py to run q-TIP4P/F water model in DASH 9/26/2018 version.  
\item Continuing to work on developing this basic test code, checking all possible options to make sure python script is correct.  
\end{enumerate}

\section{10/3/2018}
\begin{enumerate}
\item We will begin by running a series of tests on q-TIP4P/F water.  We will use the following system: 1000 water molecules, 1 ns, initial density of 0.997, temperature of 298 K, pressure 1 atm, printing data every 1000 steps.  We are using run\_9-26-2018.sh and tip4pF\_9-26-2018.py to run DASH version from 9/26/2018.  
\item On /home/swansonk1, Midway 1, go into openmm folder.  On depablo-gpu, module load Anaconda3, source activate openmm-env.  Downloaded pimd.py from Markland, called it pimd\_original.py, and created new file pimd\_modified.py.  In the modified file, we change the nonbondedCutoff to 0.9*nanometers, comment out the platform and properties assignments (so that we use OpenCL by default), and change the Simulation() to get rid of platform and properties arguments.  
\item Searching for OpenMM code here: http://docs.openmm.org/development/api-python/
\item Found RPMDMonteCarloBarostat.  "This class is very similar to MonteCarloBarostat, but it is specifically designed for use with RPMDIntegrator.  For each trial move, it scales all copies of the system by the same amount, then accepts or rejects the move based on the change to the total energy of the ring polymer."
\item Having trouble understanding the kinetic energy units in DASH...not sure what is going on here.  I just want to compute the total kinetic energy.  For now we will leave it out then.  
\subitem Tests using the Isotropic Monte Carlo Barostat:
\subsubitem Time Step: 0.5 fs, rCut: 9 A, nBeads: 1, jobname MC-1/B-1/MM-1
\subsubitem Time Step: 0.5, rCut: 9, nBeads: 16, jobname MC-2/B-2/MM-2
\subsubitem Time Step: 0.5, rCut: 9, nBeads: 32, jobname MC-3/B-3/MM-3
\subsubitem Time Step: 0.5, rCut: 12, nBeads: 1, jobname MC-4/B-4/MM-4
\subsubitem Time Step: 0.5, rCut: 12, nBeads: 16, jobname MC-5/B-5/MM-5
\subsubitem Time Step: 0.5, rCut: 12, nBeads: 32, jobname MC-6/B-6/MM-6
\subsubitem Time Step: 0.25, rCut: 9, nBeads: 1, jobname MC-7/B-7/MM-7
\subsubitem Time Step: 0.25, rCut: 9, nBeads: 16, jobname MC-8/B-8/MM-8
\subsubitem Time Step: 0.25, rCut: 9, nBeads: 32, jobname MC-9/B-9/MM-9
\subsubitem Time Step: 1.0, rCut: 9, nBeads: 1, jobname MC-10/B-10/MM-10
\subsubitem Time Step: 1.0, rCut: 9, nBeads: 16, jobname MC-11/B-11/MM-11
\subsubitem Time Step: 1.0, rCut: 9, nBeads: 32, jobname MC-12/B-12/MM-12
\subsubitem Time Step: 0.5 fs, rCut: 9A, nBeads: 1, NO BAROSTAT, jobname dash-noP/open-noP (compare energies at same density)
\subsubitem Time Step: 0.5 fs, rCut: 9A, nBeads: 64, jobname MC-13/B-13/MM-13
\subsubitem Time Step: 0.5 fs, rCut: 9A, nBeads: 128, jobname MC-13/B-13/MM-13
\end{enumerate}

\section{10/4/2018}
\begin{enumerate}
\item MC-1 thru 3 and MM-1 thru 3 finished running.  Let's now analyze the results.  
\item Wrote data\_analysis.py in Path Integrals/hexane-water/dash\_work/water folder for analysis of this data.  
\item Here is an initial result for Time Step: 0.5 fs, rCut: 9 A, nBeads: 1:
\begin{figure}[H]
\centering
\includegraphics[scale=0.4]{MC-MM-nBead1-rCut9-ts05}
\end{figure}
\item Realized that we should have an equilibration period before comparing averages, since the beginning of the simulation seems to screw things up.  We should also report average values with SE values.  
\item Using simulations MC-1 and MM-1, we can look at the standard error of the density for various block sizes to choose an appropriate number of blocks for block averaging.  Wrote the file SE\_analysis.py in the same folder as above.  We get the following:
\begin{figure}[H]
\centering
\includegraphics[scale=0.4]{block_averaging_DASH-vs-OpenMM}
\end{figure}
\item Based on this figure, we will choose 100 ps for an appropriate block size.  Since we are reserving the first 100 ps for equilibration, this gives us 9 blocks of size 100 ps.  
\item We also tried a different format allowing for variable block sizes:
\begin{figure}[H]
\centering
\includegraphics[scale=0.4]{block_averaging_DASH-vs-OpenMM_variable}
\end{figure}
\item The If the number of blocks is 2.5, then it leaves out the last 0.5 of data.  There could be other ways of doing this...but for now, this is fine.  Result is still the same: 100 ps is a good block size.  
\item Using 9 blocks of size 100 ps, we then have (USING +/- 2*SE):
\subitem DASH Density: 0.98974 $\pm$ 0.00072 g/mL
\subitem OpenMM Density: 1.00073 $\pm$ 0.00132 g/mL
\subitem DASH Potential Energy: -42276.85961 $\pm$ 35.38552 kJ/mol
\subitem OpenMM Potential Energy: -43041.14116 $\pm$ 33.45556 kJ/mol
\subitem DASH Temperature: 300.30567 $\pm$ 0.27942 K
\subitem OpenMM Temperature: 300.09525 $\pm$ 0.386 K
\begin{figure}[H]
\centering
\includegraphics[scale=0.7]{MC-MM-nBead1-rCut9-ts05-new}
\end{figure}
\item For Time Step: 0.5 fs, rCut: 9 A, nBeads 16:
\item Using 9 blocks of size 100 ps, we then have:
\subitem DASH Density: 1.01268 $\pm$ 0.00176 g/mL
\subitem OpenMM Density: 0.99751 $\pm$ 0.00104 g/mL
\subitem DASH Temperature: 4774.04631 $\pm$ 1.91432 K
\subitem OpenMM Temperature: 4788.12303 $\pm$ 2.5978 K
\begin{figure}[H]
\centering
\includegraphics[scale=0.7]{MC-MM-nBead16-rCut9-ts05}
\end{figure}

\item For Time Step: 0.5 fs, rCut: 9 A, nBeads 32:
\item Using 9 blocks of size 100 ps, we then have:
\subitem DASH Density: 1.03649 $\pm$ 0.00082 g/mL
\subitem OpenMM Density: 0.9995 $\pm$ 0.00176 g/mL
\subitem DASH Temperature: 9532.01452 $\pm$ 2.57316 K
\subitem OpenMM Temperature: 9555.22163 $\pm$ 6.96436 K
\begin{figure}[H]
\centering
\includegraphics[scale=0.7]{MC-MM-nBead32-rCut9-ts05}
\end{figure}

\item It would be interesting to see what happens for 64 beads.  Let's run both of these under the same above conditions.  These will be simulations no. 23 and 24.  
\item We also want to run the Berendsen barostat in DASH for 0.5 fs, rCut 9, and nBeads 1, 16, and 32.  These will be simulations 25, 26, 27.  
\item It would also be interesting to run with no PI beads for 0.5 fs, rCut 9, which will be simulation 28 and 29 for Berendsen and MonteCarlo, respectively.
\item We also want to run the simulations that compare different rCut, so we will run DASH MC barostat using 0.5 fs, rCut 12 A, and nBeads 1, 16, 32, which will be simulations 30, 31, 32 and job names MC-4, MC-5, MC-6.  We will also do this in OpenMM, jobs MM-4, MM-5, MM-6 and simulations 33, 34, 35.  
\end{enumerate}

\section{10/5/2018}
\begin{enumerate}
\item For time step 0.5 fs, rCut 9 A, nBeads 1, using the Berendsen barostat for DASH:
\subitem DASH Density: 0.98832 $\pm$ 0.00058 g/mL
\subitem OpenMM Density: 1.00073$\pm$ 0.00132 g/mL
\subitem DASH Potential Energy: -42320.03482 $\pm$  35.48798 kJ/mol
\subitem OpenMM Potential Energy: -43041.14116 $\pm$ 33.45556 kJ/mol
\subitem DASH Temperature: 299.70865 $\pm$ 0.45264  K
\subitem OpenMM Temperature:  300.09525 $\pm$ 0.386 K
\begin{figure}[H]
\centering
\includegraphics[scale=0.7]{B-MM-nBead1-rCut9-ts05}
\end{figure}
\item For time step 0.5 fs, rCut 9 A, no PI beads for DASH version and 1 for OpenMM, using Berendsen for DASH:
\subitem DASH Density: 0.98821 $\pm$ 0.00052 g/mL
\subitem OpenMM Density: 1.00073$\pm$ 0.00132 g/mL
\subitem DASH Potential Energy: -42341.16144 $\pm$  41.09194 kJ/mol
\subitem OpenMM Potential Energy: -43041.14116 $\pm$ 33.45556 kJ/mol
\subitem DASH Temperature: 299.77902 $\pm$ 0.6386  K
\subitem OpenMM Temperature:  300.09525 $\pm$ 0.386 K
\begin{figure}[H]
\centering
\includegraphics[scale=0.7]{B-no-PI-MM-nBead1-rCut9-ts05}
\end{figure}

\item For time step 0.5 fs, rCut 9 A, no PI beads for DASH version and 1 for OpenMM, using MonteCarlo for DASH:
\subitem DASH Density: 0.98886 $\pm$ 0.00052 g/mL
\subitem OpenMM Density: 1.00073$\pm$ 0.00116 g/mL
\subitem DASH Potential Energy: -42286.8873 $\pm$  29.45386 kJ/mol
\subitem OpenMM Potential Energy: -43041.14116 $\pm$ 33.45556 kJ/mol
\subitem DASH Temperature: 300.28388 $\pm$ 0.34328  K
\subitem OpenMM Temperature:  300.09525 $\pm$ 0.386 K
\begin{figure}[H]
\centering
\includegraphics[scale=0.7]{MC-no-PI-MM-nBead1-rCut9-ts05}
\end{figure}

\item For time step 0.5 fs, rCut 12 A, nBeads 1 using MonteCarlo for DASH:
\subitem DASH Density: 0.99837 $\pm$ 0.0012 g/mL
\subitem OpenMM Density: 1.00162$\pm$ 0.00092 g/mL
\subitem DASH Potential Energy: -42634.1723 $\pm$ 34.76596 kJ/mol
\subitem OpenMM Potential Energy: -43037.889 $\pm$ 22.094 kJ/mol
\subitem DASH Temperature: 300.37583 $\pm$ 0.39688  K
\subitem OpenMM Temperature:  300.22983 $\pm$ 0.526 K
\begin{figure}[H]
\centering
\includegraphics[scale=0.7]{MC-MM-nBead1-rCut12-ts05}
\end{figure}

\item For time step 0.5 fs, rCut 12 A, nBeads 16 using MonteCarlo for DASH:
\subitem DASH Density: 1.01984 $\pm$ 0.00176 g/mL
\subitem OpenMM Density: 0.99702$\pm$ 0.00106 g/mL
\subitem DASH Temperature: 4773.6277 $\pm$ 1.52266  K
\subitem OpenMM Temperature:  4785.29712 $\pm$ 4.35954 K
\begin{figure}[H]
\centering
\includegraphics[scale=0.7]{MC-MM-nBead16-rCut12-ts05}
\end{figure}

\item For time step 0.5 fs, rCut 12 A, nBeads 32 using MonteCarlo for DASH:
\subitem DASH Density: 1.04464 $\pm$ 0.00154 g/mL
\subitem OpenMM Density: 0.99902 $\pm$ 0.0005 g/mL
\subitem DASH Temperature: 9531.9981 $\pm$ 3.0  K
\subitem OpenMM Temperature:  9549.73486 $\pm$ 7.4266 K
\begin{figure}[H]
\centering
\includegraphics[scale=0.7]{MC-MM-nBead32-rCut12-ts05}
\end{figure}

\item Now we will run MC/MM-7, 8, 9, 10, 11, 12 as specified above, which will be simulations 36-47. 
\item Now, we will try running an NVT system to see if there is still an energy difference in the 1 bead case.  To do this, we want to make sure we are running the DASH and OpenMM systems at the same number of molecules and densities.  So, let's see what the default OpenMM density is.
\item Running an NVT test, we see that the density is exactly 0.9979734044480487.  So, let's run DASH using the standard parameters in NVT with 1 bead and this exact initial density, and same thing with OpenMM, and compare energies.  These will be simulations 48 and 49.   
\item We also want to check the standard NPT run increasing rCut to 15: for DASH and OpenMM, these will be simulations 50 and 51.  


\item NVT same density/number molecules for 1 bead with charge, then NPT and NVT for 1 bead turning charges off
\item Try doing another dummy system: van der Waals particles/LJ particles in a box, same test as above 
\item Rule out: problem related to long-range electrostatics, is that problem specific to tip4p/F, could try tip3p or something.  Paesani and Voth SPC perhaps
\item Check all parameters of the OpenMM model for accuracy...and comparison to DASH...
\item check 1-4 interactions??? how does this work in DASH?  what is the default nonbonded 1-4 interaction in dash?
\end{enumerate}

\section{10/6/2018}
\begin{enumerate}
\item Let's start by constructing water models that have zero electrostatic charge.  
\item For OpenMM, creating a file called qtip4pf-0charge.xml where we will set the charge for the H atoms and M-site to zero.  
\item For DASH, created a new addition to the water.py utils\_py script called create\_TIP4PFlexible\_0charge, which creates water molecules with zero charge.  We will now run simulations 52 and 53, which will be DASH MonteCarlo NPT with standard stuff but zero charge and OpenMM zero charge
\item Looks like sims. 51 and 52 did not equilibrate, so let's run for 10 ns and see if that helps.  This will be 54 adn 55.  
\end{enumerate}

\section{10/7/2018}
\begin{enumerate}
\item Let's start by creating the full story of plots with things that we've done so far. 
\item Updated plot for MM-1 and MC-1:
\begin{figure}[H]
\centering
\includegraphics[scale=0.7]{MC-MM-nBead1-rCut9-ts05-update}
\end{figure}
\item Leaving out energies for multiple beads b/c DASH reports energies on the order of 2M kJ/mol while OpenMM reports on the order of -20,000 kJ/mol.  
\item Updated plot for MC-2 and MM-2: 
\begin{figure}[H]
\centering
\includegraphics[scale=0.7]{MC-MM-nBead16-rCut9-ts05-update}
\end{figure}
\item Updated plot for MC-3 and MM-3: 
\begin{figure}[H]
\centering
\includegraphics[scale=0.7]{MC-MM-nBead32-rCut9-ts05-update}
\end{figure}
\item For time step 0.5 fs, rCut 9 A, nBeads 1 using Berendsen for DASH:
\begin{figure}[H]
\centering
\includegraphics[scale=0.7]{B-MM-nBead1-rCut9-ts05-update}
\end{figure}

\item For time step 0.5 fs, rCut 9 A, nBeads 16 using Berendsen for DASH: something is clearly erroneous:
The density for the DASH version quickly drops off close to zero...so something weird is clearly happening.  Same thing with the 32 bead version...

\item For time step 0.5 fs, rCut 12 A, nBeads 1 using MonteCarlo for DASH (data copied from above):
\subitem DASH Density: 0.99837 $\pm$ 0.0012 g/mL
\subitem OpenMM Density: 1.00162$\pm$ 0.00092 g/mL
\subitem DASH Potential Energy: -42634.1723 $\pm$ 34.76596 kJ/mol
\subitem OpenMM Potential Energy: -43037.889 $\pm$ 22.094 kJ/mol
\subitem DASH Temperature: 300.37583 $\pm$ 0.39688  K
\subitem OpenMM Temperature:  300.22983 $\pm$ 0.526 K
\begin{figure}[H]
\centering
\includegraphics[scale=0.7]{MC-MM-nBead1-rCut12-ts05-update}
\end{figure}

\item For time step 0.5 fs, rCut 12 A, nBeads 16 using MonteCarlo for DASH (data copied from above):
\subitem DASH Density: 1.01984 $\pm$ 0.00176 g/mL
\subitem OpenMM Density: 0.99702$\pm$ 0.00106 g/mL
\subitem DASH Temperature: 4773.6277 $\pm$ 1.52266  K
\subitem OpenMM Temperature:  4785.29712 $\pm$ 4.35954 K
\begin{figure}[H]
\centering
\includegraphics[scale=0.7]{MC-MM-nBead16-rCut12-ts05-update}
\end{figure}

\item For time step 0.5 fs, rCut 12 A, nBeads 32 using MonteCarlo for DASH (data copied from above):
\subitem DASH Density: 1.04464 $\pm$ 0.00154 g/mL
\subitem OpenMM Density: 0.99902 $\pm$ 0.0005 g/mL
\subitem DASH Temperature: 9531.9981 $\pm$ 3.0  K
\subitem OpenMM Temperature:  9549.73486 $\pm$ 7.4266 K
\begin{figure}[H]
\centering
\includegraphics[scale=0.7]{MC-MM-nBead32-rCut12-ts05-update}
\end{figure}

\item For time step 0.25 fs, rCut 9 A, nBeads 1 using MonteCarlo for DASH:
\subitem DASH Density: 0.98814 $\pm$ 0.00115 g/mL
\subitem OpenMM Density: 1.0007 $\pm$ 0.0008 g/mL
\subitem DASH Potential Energy: -42444.13902 $\pm$ 35.19276 kJ/mol
\subitem OpenMM Potential Energy: -43114.43317 $\pm$ 29.3608 kJ/mol
\subitem DASH Temperature: 298.5529 $\pm$ 0.3806  K
\subitem OpenMM Temperature:  298.44123 $\pm$ 0.26228 K
\begin{figure}[H]
\centering
\includegraphics[scale=0.7]{MC-MM-nBead1-rCut9-ts025}
\end{figure}

\item For time step 0.25 fs, rCut 9 A, nBeads 16 using MonteCarlo for DASH:
\subitem DASH Density: 0.98032 $\pm$ 0.0011 g/mL
\subitem OpenMM Density: 0.99773 $\pm$ 0.001188 g/mL
\subitem DASH Temperature: 297.083 $\pm$ 0.0226  K
\subitem OpenMM Temperature:  298.3412 $\pm$ 0.1347 K
\begin{figure}[H]
\centering
\includegraphics[scale=0.7]{MC-MM-nBead16-rCut9-ts025}
\end{figure}

\item For time step 0.25 fs, rCut 9 A, nBeads 32 using MonteCarlo for DASH:
\subitem DASH Density: 0.97463 $\pm$ 0.00156 g/mL
\subitem OpenMM Density: 0.99888 $\pm$ 0.0011 g/mL
\subitem DASH Temperature: 296.85178 $\pm$ 0.078387  K
\subitem OpenMM Temperature:  298.212148 $\pm$ 0.16234 K
\begin{figure}[H]
\centering
\includegraphics[scale=0.7]{MC-MM-nBead32-rCut9-ts025}
\end{figure}

\item For time step 1.0 fs, rCut 9 A, nBeads 1 using MonteCarlo for DASH:
\subitem DASH Density: 0.99712 $\pm$ 0.00159 g/mL
\subitem OpenMM Density: 0.99966 $\pm$ 0.00214 g/mL
\subitem DASH Potential Energy: -41438.529 $\pm$ 42.6877 kJ/mol
\subitem OpenMM Potential Energy: -42710.35912 $\pm$ 32.50538 kJ/mol
\subitem DASH Temperature: 309.13899 $\pm$ 0.43108  K
\subitem OpenMM Temperature:  307.90749 $\pm$ 0.438 K
\begin{figure}[H]
\centering
\includegraphics[scale=0.7]{MC-MM-nBead1-rCut9-ts1}
\end{figure}

\item For time step 1.0 fs, rCut 9 A, nBeads 16 using MonteCarlo for DASH:
\subitem DASH Density:1.18474 $\pm$ 0.0009 g/mL
\subitem OpenMM Density: 0.996576 $\pm$ 0.00175 g/mL
\subitem DASH Temperature: 313.89341 $\pm$ 0.14455  K
\subitem OpenMM Temperature:  303.714367 $\pm$ 0.30905 K
\begin{figure}[H]
\centering
\includegraphics[scale=0.7]{MC-MM-nBead16-rCut9-ts1}
\end{figure}


\end{enumerate}

\section{10/8/2018}
\begin{enumerate}
\item Compare 1 bead (and possibly multiple bead?) cases to LAMMPS, so we can see what the energy should be etc.  
\item Do multiple beads in NVT
\item Understand energy completely, diagnose the NVT case first before NPT
\item Figure out energy issue, like what is the comparable energy and what are DASH and OpenMM actually computing (maybe restrict DASH to specific fixes?)
\item Try SPC water model to eliminate M-site problem
\item Run 0 charge for 20M steps instead of just 10M to get better equilibration convergence 
\item Try potential from Martyna 1999 to test basic PIMD integrator/barostat 
\item For time step 1.0 fs, rCut 9 A, nBeads 32 using MonteCarlo for DASH:
\subitem DASH Density: Over 190,000, dropped from 0.997 to 0.855.  
\subitem OpenMM Density: 0.99617 $\pm$ 0.00085 g/mL
\subitem DASH Temperature: Temperature and Energy failed to record (nans), and .xyz file filled with nans as well.  
\subitem OpenMM Temperature:  301.93122 $\pm$ 7.58815 K
\item The above problem appeared to persist from time step 0.9 fs and higher. 

\item For time step 0.5 fs, rCut 9 A, nBeads 1 using NVT for DASH and OpenMM:
\subitem DASH Potential Energy: -42320.31975 $\pm$ 24.91336 kJ/mol
\subitem OpenMM Potential Energy: -43040.87081 $\pm$ 20.17038 kJ/mol
\subitem DASH Temperature: 300.2369 $\pm$ 0.32871  K
\subitem OpenMM Temperature: 300.3045 $\pm$ 0.25418 K
\begin{figure}[H]
\centering
\includegraphics[scale=0.7]{MC-MM-NVT-1bead}
\end{figure} 

\item For time step 0.5 fs, rCut 9 A, nBeads 1 DASH Monte Carlo, 0 charge for both (simulations 54 and 55), using the entire dataset, 6 blocks for SE analysis :
\subitem DASH Density: 0.049276 $\pm$ 0.055271 g/mL
\subitem OpenMM Density: 0.042175 $\pm$ 0.062486 g/mL
\subitem DASH Potential Energy: 3706.58614 $\pm$ 222.6355 kJ/mol
\subitem OpenMM Potential Energy: 3635.81484 $\pm$ 289.05871 kJ/mol
\subitem DASH Temperature: 300.54083 $\pm$ 0.19621  K
\subitem OpenMM Temperature:  300.42003 $\pm$ 0.19215 K
\begin{figure}[H]
\centering
\includegraphics[scale=0.7]{MC-MM-0charge}
\end{figure} 
\begin{figure}[H]
\centering
\includegraphics[scale=0.7]{block_averaging_0charge}
\end{figure} 

\item Using steps 6000 thru 10000 from SE analysis, chose 4 blocks of size 1000 ps 
\subitem DASH Potential Energy: 3878.93275 $\pm$ 18.3898 kJ/mol
\subitem OpenMM Potential Energy: 3930.67033 $\pm$ 6.60727 kJ/mol
\subitem DASH Temperature: 300.4556 $\pm$ 0.37655  K
\subitem OpenMM Temperature:  300.36912 $\pm$ 0.190716 K
\begin{figure}[H]
\centering
\includegraphics[scale=0.7]{MC-MM-0charge-narrow}
\end{figure} 

\item For time step 0.5 fs, rCut 9 A, nBeads 1, NVT, 0 charge (simulations 56 and 57):
Using steps 0 thru 10000 from SE analysis, chose 10 blocks of size 1000 ps 
\subitem DASH Potential Energy: 1787.64728 $\pm$ 3.91321 kJ/mol
\subitem OpenMM Potential Energy: 1124.83421 $\pm$ 5.52062 kJ/mol
\subitem DASH Temperature: 300.41357 $\pm$ 0.20056  K
\subitem OpenMM Temperature:  300.54907 $\pm$ 0.20918 K
\begin{figure}[H]
\centering
\includegraphics[scale=0.7]{MC-MM-0charge-NVT}
\end{figure} 
\end{enumerate}















\end{document}