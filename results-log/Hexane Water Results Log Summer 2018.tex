% LaTEX source code
% Last modified November 1st, 2005
% Steve Miller
% note that the percent sign comments out the rest of the line
% first, we set a document class. often use 12pt characters, though
% sometimes people do 11 or 10. you can do report or article, both similar
%\documentclass[12pt,letterpaper]{article}
\documentclass[12pt,reqno]{amsart}
\linespread{1}
\addtolength{\textwidth}{2cm} \addtolength{\hoffset}{-1cm}
\addtolength{\marginparwidth}{-1cm} \addtolength{\textheight}{2cm}
\addtolength{\voffset}{-1cm}
% below are some packages that are needed for certain symbols, graphics, colors.
% safest to just include these.
\usepackage{times}
\usepackage[T1]{fontenc}
\usepackage{mathrsfs}
\usepackage{latexsym}
\usepackage[dvips]{graphics}
\usepackage{epsfig}
\usepackage{hyperref, amsmath, amsthm, amsfonts, amscd, flafter,epsf}
\usepackage{amsmath,amsfonts,amsthm,amssymb,amscd}
\input amssym.def
\input amssym.tex
\usepackage{color}
\usepackage{enumerate}
\usepackage{hyperref}
\usepackage{url}
\usepackage{floatrow}
\usepackage{caption}
\usepackage{subcaption}
\usepackage{capt-of}
\usepackage{physics}
\newcommand{\todo}[1]{\textcolor{red}{\textbf{(#1)}}}

    %=======================================================

    %   THIS IS WHERE YOU PUT SHORTCUT DEFINITIONS

    %========================================================

% Note that we use a percent sign to comment out a line

% below are shortcut commands

%%%%%%%%%%%%%%%%%%%%%%%%%%%%%%%%%%%%%%%%%%%%%%%

% below are shortcuts for equation, eqnarray,

% itemize and enumerate environments

\newcommand\be{\begin{equation}}
\newcommand\ee{\end{equation}}
\newcommand\bea{\begin{eqnarray}}
\newcommand\eea{\end{eqnarray}}
\newcommand\bi{\begin{itemize}}
\newcommand\ei{\end{itemize}}
\newcommand\ben{\begin{enumerate}}
\newcommand\een{\end{enumerate}}
\newcommand{\ncr}[2]{\left({#1 \atop #2}\right)}
%%%%%%%%%%%%%%%%%%%%%%%%%%%%%%%%%%%%%%%%%%%%%%%%

% Theorem / Lemmas et cetera

\newtheorem{thm}{Theorem}[section]
\newtheorem{conj}[thm]{Conjecture}
\newtheorem{cor}[thm]{Corollary}
\newtheorem{lem}[thm]{Lemma}
\newtheorem{prop}[thm]{Proposition}
\newtheorem{exa}[thm]{Example}
\newtheorem{defi}[thm]{Definition}
\newtheorem{exe}[thm]{Exercise}
\newtheorem{rek}[thm]{Remark}
\newtheorem{que}[thm]{Question}
\newtheorem{prob}[thm]{Problem}
\newtheorem{cla}[thm]{Claim}
\newtheorem{defis}[thm]{Definitions}
\newtheorem{res}[thm]{Result}
\newtheorem{calc}[thm]{Calculation}
%%%%%%%%%%%%%%%%%%%%%%%%%%%%%%%%%%%%%%%%%

% shortcuts to environments

% this allows you to do textboldface: simply type \tbf{what you want in bold}

\newcommand{\tbf}[1]{\textbf{#1}}

%%%%%%%%%%%%%%%%%%%%%%%%%%%%%%%%%%%%%%%%%%%%%%%%%%

% shortcut to twocase and threecase definitions

\newcommand{\twocase}[5]{#1 \begin{cases} #2 & \text{#3}\\ #4
&\text{#5} \end{cases}   }
\newcommand{\threecase}[7]{#1 \begin{cases} #2 &
\text{#3}\\ #4 &\text{#5}\\ #6 &\text{#7} \end{cases}   }
%%%%%%%%%%%%%%%%%%%%%%%%%%%%%%%%%%%%%%%%%

%Blackboard Letters

\newcommand{\R}{\ensuremath{\mathbb{R}}}
\newcommand{\C}{\ensuremath{\mathbb{C}}}
\newcommand{\Z}{\ensuremath{\mathbb{Z}}}
\newcommand{\Q}{\mathbb{Q}}
\newcommand{\N}{\mathbb{N}}
\newcommand{\F}{\mathbb{F}}
\newcommand{\W}{\mathbb{W}}
\newcommand{\Qoft}{\mathbb{Q}(t)}  %use in linux
\newcommand{\soln}{\noindent \textbf{Solution:}\ }

%%%%%%%%%%%%%%%%%%%%%%%%%%%%%%%%%%%%%%%%%

% Finite Fields and Groups

\newcommand{\Fp}{ \F_p }
%%%%%%%%%%%%%%%%%%%%%%%%%%%%%%%%%%%%%%%%%

% Fractions

\newcommand{\foh}{\frac{1}{2}}  %onehalf
\newcommand{\fot}{\frac{1}{3}}
\newcommand{\fof}{\frac{1}{4}}

%%%%%%%%%%%%%%%%%%%%%%%%%%%%%%%%%%%%%%%%%

% Legendre Symbols

\newcommand{\js}[1]{ { \underline{#1} \choose p} }

%%%%%%%%%%%%%%%%%%%%%%%%%%%%%%%%%%%%%%%%%

% matrix shortcuts

\newcommand{\mattwo}[4]
{\left(\begin{array}{cc}
                        #1  & #2   \\
                        #3 &  #4
                          \end{array}\right) }
\newcommand{\matthree}[9]
{\left(\begin{array}{ccc}
                        #1  & #2 & #3  \\
                        #4 &  #5 & #6 \\
                        #7 &  #8 & #9
                          \end{array}\right) }
\newcommand{\dettwo}[4]
{\left|\begin{array}{cc}
                        #1  & #2   \\
                        #3 &  #4
                          \end{array}\right| }
\newcommand{\detthree}[9]
{\left|\begin{array}{ccc}
                        #1  & #2 & #3  \\
                        #4 &  #5 & #6 \\
                        #7 &  #8 & #9
                          \end{array}\right| }
%%%%%%%%%%%%%%%%%%%%%%%%%%%%%%%%%%%%%%%%%

% greek letter shortcuts

\newcommand{\ga}{\alpha}                  %gives you a greek alpha
\newcommand{\gb}{\beta}
\newcommand{\gep}{\epsilon}
%%%%%%%%%%%%%%%%%%%%%%%%%%%%%%%%%%%%%%%%%

% general functions

\newcommand{\notdiv}{\nmid}               % gives the not divide symbol
\newcommand{\burl}[1]{\textcolor{blue}{\url{#1}}}

%%%%%%%%%%%%%%%%%%%%%%%%%%%%%%%%%%%%%%%%%%%

% the following makes the numbering start with 1 in each section;

% if you want the equations numbered 1 to N (without caring about

% what section you are in, comment out the following line.

\numberwithin{equation}{section}

%\textwidth= 6in

%\evensidemargin=37pt

%\oddsidemargin=0pt

\begin{document}



\title{Hexane Water Results Log Summer 2018}
\author{Kirk Swanson}
\email{swansonk1@uchicago.edu}
\address{Institute for Molecular Engineering, University of Chicago, 5640 S Ellis Ave, Chicago, IL 60637}
\date{\today}




\maketitle

%%%%%%%%%%%%%%%%%%%%%%%%%%%%%%%%%%%%%%%%%%%%%%%%%%%%%%%%%%%%%%%%%%%%%%%%%%%%%%%%%%%%%%%%%%%%%%%%%%%%%%%%%%%%%%%%%%%%%%%%%%%%%%

\normalsize

%%%%%%%%%%%%%%%%%%%%%%%%%%%%%%%%%%%%%%%%%%%%%%%%%%%%%%%%%%%%%%%%%%%%%%%%%%%%%%%%%%%%%%%%%%%%%%%%%%%%%%%%%%%%%%%%%%%%%%%%%%%%%%
\section{8/6/2018}
\begin{enumerate}
\item Downloaded pressure\_scalar-press-NPT-1bead-water-lammps to dash\_work/interface on laptop.  Used pressure\_analysis\_UPDATE.py leaving out first 2000 records and used 198 blocks 5 ps to get average pressure of 1.41, SE 0.94, SD 511.08: 
\begin{figure}[H]
\centering
\includegraphics[scale=0.4]{pressures_press-NPT-1bead-water-lammps}
\end{figure}
\item Because we accidentally used restart/traj every 1,000 for fulldata-NPT-298 and then restart/traj 10,000 for fulldata-NPT-298-restart1, we will re-do this calculation again using the updated interface file collecting restart/traj every 1,000.  
\item Submitted simulation no. 1.  
\item To check pressures in a non-vacuum interfacial system, in DASH we are going to run simulation no. 2, which does NPT equilibration for 200,000 steps and then does an NVT simulation of the interface.  Filename press-nonexpanded.
\item Submitted simulation no. 2.  
\item We want to compare the above simulation to LAMMPS, so we will first run the equilibration step in LAMMPS as simulation no. 3.  Note, however, that the mixing rules currently used in the LAMMPS file are arithmetic.  Also NOTE: to use depablo-tc, must type module load openmpi into terminal first, even if it is already in submit script.  After this, we need to run the NVT portion of the simulation for 6M steps.  
\item Running simulation no. 3.
\item Now, let's turn to how pressure is typically computed in DASH.  In general, an ideal gas mechanical equation of state is given by 
\begin{align}
\begin{split}
P = \frac{Nk_BT}{V} = \rho k_BT
\end{split}
\end{align}
This assumes that molecules in the gas are point-like and do not interact with each other.  However, if interactions between the particles and a finite particle volume is allowed.  The virial expansion is a perturbation theory around the ideal gas used when the interactions in the real gas are dominated by two-body interactions.  One uses a virial expansion in powers of the density, keeping the temperature dependence of the coefficients.  We can then apply manipulations of the time derivative of the classical virial to arrive at 
\begin{align}
\begin{split}
P = \rho k_B T - \left[\frac{1}{3V}\left<\sum_{i<j}^Nr_{ij}F_{ij}\right>\right]
\end{split}
\end{align}
where the second term is called the virial correction.  
\item In DASH, it appears as though DataComputerPressure.cu is responsible for scalar pressure values.  On line 92, we have 
\begin{align}
\begin{split}
pressureScalar = (tempScalar\_loc * ndf\_loc * boltz + sumVirial) / (dim * volume) * state->units.nktv\_to\_press
\end{split}
\end{align}
tempScalar\_loc appears to be the temperature of the system, boltz is the boltzmann coefficient, sumVirial is that second virial correction, dim is the dimension of the system.  ndf I calculated to be 3 per atom, including M sites for the water model (printed values using interface file).  So, it does make sense to divide by dim, i.e. divide by 3, because that gets us to N.  However, we are also including M-sites in the pressure value.  So, we are at the very least overcounting the ideal gas term for the pressure.  The amount of overcorrection for 3650 water molecules is given by:
\begin{align}
\begin{split}
P_{c} = \frac{Nk_BT}{V} = \frac{3650\times1.38\times10^{-23}\times 298}{50 \times 10^{-10}\times 50\times 10^{-10}\times 150\times 10^{-10}}
\end{split}
\end{align}
Actually, would it even be incorrect to have four sites per atom?  Maybe that's actually the correct ideal model here?
\end{enumerate}

\section{8/7/2018}
\begin{enumerate}
\item First, we will begin by compiling the most recent version of DASH, which should contain Mike's edits to the TIP4P/F fix for accurate pressure computation (avoiding M-site in ideal gas term).  Compiling version in folder DASH-8-7-2018.  
\item DASH pressure computations:
\subitem NVT with vacuum: run\_UPDATE\_8-7-2018.sh and interface\_UPDATE\_8-7-2018.py, z length 149.2717, x length 58.45467, y length 58.67825, 3650 TIP4P/F water molecules, 500 hexane molecules, waldman-hagler mixing between the water and hexane molecules, NVT with Andersen thermostat with parameters nu = 0.01 (a parameter describing the collision frequency of the system with the heat bath) and applying every 10 steps, 298 K, 1M steps equilibration and 1M steps production, PI false, restart/traj every 10,000, -20/+20 for z change as accounted above, filename newpress-expanded, tensor information commented out in input script. This is simulation no. 4.   
\subitem We will do the same as above, except we will not expand the box by 20 A in each z direction, and we will apply NPT equilibration for 1M steps and then 1M steps of NVT.  This is simulation no. 6.  

\item LAMMPS pressure computations:
\subitem NVT with vacuum: interface.sbatch and in.taffi\_tip4pF\_waldman, x length 149.2717, y length 58.67825, z length 58.45467, 3650 TIP4P/F water molcules, 500 hexane molecules, waldman-hagler mixing, NVT with NoseHoover, 298 K, 1M steps equilibration and 1M steps production, PI false, restart/traj every 10,000, filename newpress-expanded-lammps.  We will start with 500,000 steps and restart from there, which is simulation no. 5. 
\subitem Same as above, except we will not expand the box by 20 A in each z direction, and we will apply NPT equilibration for 1M steps and then 1M steps of NVT.  The first 500,000 steps will be simulation no. 7.  
\subitem Simulation no. 8 is filename newpress-expanded-lammps-restart1, restarting from newpress-expanded-lammps.restart for 1.5M steps 
\subitem Simulation no. 9 is filename newpress-nonexpanded-lammps-restart1, restarting from newpress-nonexpanded-lammps.restart for 500K steps
\subitem Simulation no. 10 is filename newpress-nonexpanded-lammps-restart2, restarting from newpress-nonexpanded-lammps-restart1.restart, and going for 1M NVT steps.  
\end{enumerate}


\end{document}